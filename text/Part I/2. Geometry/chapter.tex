\section{Grassmannian}
Let $K$ be an algebraically closed field. In this section, we will develop the concept of Grassmannian, which gives us a way to find the $27$ lines.

\begin{defn}
	The Grassmannian variety $G(k,n)$ for $0\le k \le n$ is defined to be\footnote{As a matter of convention, some mathematicians will write it as $G(n,k)$.}
	\[
		G(n,k)=\{V \text{ subspace of } K^n:\dim V = k\}.
	\]
\end{defn}

In the work of $27$ lines, we will work with $G(2,4)$, which denotes the (projective) lines over the (projective) space. In order to make $G(k,n)$ a \textit{de facto} variety, we will introduce thee Plücker coordinate system. The idea of the coordinate system is simple: every $U \in G(k,n)$ admits a basis consisting of $k$ vectors $\{u_1,\dots,u_k\}$ of dimension $n$, thus we have a $n \times k$ matrix whose columns are $u_1,\dots,u_d$. We also want to pay respect to the order of vectors in the basis chosen. A tool that can track the order of vectors is the determinant function, who is generalized by wedge product.

\begin{defn}
	The Plücker map $p:G(k,n) \to \mathbb{P}(\bigwedge^d K^n)$ is defined as follows. Let $U \in G(k,n)$ be an element with basis $\{u_1,\dots,u_k\}$, then $p(U)=[u_1 \wedge \cdots u_k]$.
\end{defn}
We need to show that it is legitimate to use the Plücker map to give a coordinate of a Grassmannian variety.
\begin{thm}\label{thm:plucker}
	Let $p:G(k,n) \to \mathbb{P}(\bigwedge^k K^n)$ be the Plücker map. Then
	\begin{enumerate}
		\item $p$ is well-defined;
		\item $p$ is injective.
	\end{enumerate}
\end{thm}
\begin{proof}
	To show that $p$ is well-defined, we need to show that $p$ does not depend on the basis chosen for an element $U \in G(k,n)$. For $U \in G(k,n)$, pick two basis $\{u_1,\dots,u_k\}$ and $\{u_1',\dots,u_k'\}$. Then each $u_i'$ can be written in the form $u_i'=\sum_{j=1}^{k}a_{ij}u_j$. Let $C=(a_{ij})$, then $C$ is invertible and the anti-commutativity of wedge product yields
	\[
		u_1'\wedge \cdots \wedge u_k'= \deg(C)u_1 \wedge \cdots \wedge u_k
	\]
	therefore $u_1'\wedge \cdots \wedge u_k'$ and $u_1 \wedge \cdots \wedge u_k$ represent the same element in $\mathbb{P}(\bigwedge^d K^n)$.
	
	
	To show that $p$ is injective, pick $U=\langle u_1,\dots,u_k\rangle$ and $V = \langle v_1,\dots,v_k\rangle$ in $G(k,n)$ such that $p(U)=p(V)$. Let $u = u_1 \wedge \cdots \wedge u_k$ and $v=v_1 \wedge \cdots \wedge v_k$, then $u = \lambda v$ for some $\lambda\in\mathbb{C}^\times$. It follows that $u_i \wedge v = u_i \wedge u = 0$ and $v_i \wedge u = v_i \wedge v = 0$ for all $i$. However the kernel of $K^n \mapsto \bigwedge^{k+1}K^n$ defined by $x \mapsto x \wedge u$ is $U$ and likewise the kernel of $x \mapsto x \wedge v$ is $V$. Therefore we must have $u_i \in V$ and $v_i \in U$ for all $i$, which shows that $U=V$.
\end{proof}

We demand two things from the Plücker map, or alternatively, from $\mathbb{P}(\bigwedge^k K^n)$. First of all, this application should allow us to do elementary calculation using elements in $\mathbb{P}(\bigwedge^k K^n)$, which can be explicitly represented. Secondly, we want to make sure that the embedding of $G(k,n)$ into $\mathbb{P}(\bigwedge^k K^n)$ makes sense in algebraic geometry. For these reasons, we define the Plücker coordinates.

To begin with, we notice that for a $K$-vector space of dimension $n$ and $0<k<n$, we have a natural isomorphism

\[
	\bigwedge^k V \cong \left(\bigwedge^{n-k}V\right)^\ast \cong \bigwedge^{n-k}V^\ast.
\]

The first identification comes from the map induced by the wedge product:

\begin{align*}
	\bigwedge^k V \times \bigwedge^{n-k}V &\to \bigwedge^n V \cong K, \\
	(x,y) &\mapsto x \wedge y.
\end{align*}

On the other hand we have $\left(\bigwedge^{n-k}V\right)^\ast \cong \bigwedge^{n-k}V^\ast$ due to the following pairing:

\begin{align*}
	\bigwedge^{n-k}V^\ast \times \bigwedge^{n-k}V &\to \bigwedge^n V \cong K, \\
	(f_1 \wedge \cdots \wedge f_{n-k},g_1\wedge \cdots \wedge g_{n-k}) &\mapsto \det(f_i(g_j)).
\end{align*}

As a matter of notation, we shall denote $\{e_1,\dots,e_n\}$ the canonical basis of $K^n$, and $\{e_1^\ast,\cdots,e_n^\ast\}$ the dual basis such that $e_i^\ast(e_j) = \delta_{ij}$, the Kronecker delta at value $(i,j)$. Let $I_{k,n}=\{\underline{i}=(i_1,\dots,i_k):1 \le i_1 <\dots<i_k \le n\}$ be the set of ordered $(k,n)$-tuples, we can then assign a basis $\{e_{\underline{i}}=e_{i_1} \wedge\cdots\wedge e_{i_k}:\underline{i}\in I_{k,n}\}$ to $\bigwedge^k V$. On the other hand, $\{p_{\underline{i}}=e_{i_1}^\ast \wedge\cdots\wedge e_{i_k}^\ast:\underline{i}\in I_{k,n}\}$ is the dual basis of $\{e_{\underline{i}}\}$ as we have

\[
	p_{\underline{i}}(e_{\underline{j}}) = \det(e_{i_\ell}^\ast(e_{j_m}))_{1 \le \ell,m \le k}=\delta_{\underline{i}\underline{j}}.
\]

The dual vectors $\{p_{\underline{i}}\}$ defines a set of projective coordinates which are the \textit{Plücker coordinates}.
